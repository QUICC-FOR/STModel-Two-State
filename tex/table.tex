% Please add the following required packages to your document preamble:
% \usepackage{booktabs}
\begin{table}[tb]
\label{tab:species_list}
\caption{List of species studied and the number of observations in the database}
\begin{tabular}{llll}
\toprule
Scientific name               & English name   & Occurrences & Transitions \\ 
\midrule
{\it Abies balsamea}          & Balsam fir        & 18961       &             \\
{\it Acer rubrum}             & Red maple         & 36726       &             \\
{\it Acer saccharum}          & Sugar maple       & 18634       &             \\
{\it Betula alleghaniensis}   & Yellow birch      & 10371       &             \\
{\it Betula papyrifera}       & Paper birch       & 15531       &             \\
{\it Fagus grandifolia}       & American beech    & 9619        &             \\
{\it Fraxinus americana}      & White ash         & 10385       &             \\
{\it Fraxinus pennsylvanica}  & Red ash           & 7662        &             \\
{\it Liriodendron tulipifera} & American sweetgum & 11045       &             \\
{\it Nyssa sylvatica}         & Blackgum          & 9209        &             \\
{\it Picea glauca}            & White spruce      & 8526        &             \\
{\it Picea mariana}           & Black spruce      &             &             \\
{\it Pinus banksiana}         & Jack pine         & 3023        &             \\
{\it Pinus strobus}           & White pine        & 8182        &             \\
{\it Pinus taeda}             & Loblolly pine     & 16351       &             \\
{\it Populus tremuloides}     & Quaking aspen     & 12736       &             \\ 
{\it Quercus alba}            & White oak         & 17280       &             \\
{\it Quercus nigra}           & Water oak         & 8291        &             \\
{\it Quercus rubra}           & Northern red oak  & 14337       &             \\
{\it Quercus velutina}        & Black oak         & 9997        &             \\
{\it Ulmus americana}         & American elm      & 11597       &             \\ 
\bottomrule
\end{tabular}
\end{table}

\begin{table}[tb]
\begin{threeparttable}
\label{tab:model_selection}
\caption{Colonization-extinction model selected for each species studied}
\begin{tabular}{llllll}
\toprule
Scientific name               & Temperature Variable    & Precipitation Variable    & $\Delta$AIC\tnote{1} & AIC$_w$\tnote{2} \\ 
\midrule
{\it Abies balsamea}          & Mean Annual Temperature & Mean Annual Precipitation & 2.15                 & 0.74             \\
{\it Acer rubrum}             & Mean Annual Temperature & Precip. Warmest Quarter   & 0.36                 & 0.27             \\
{\it Acer saccharum}          & Mean Annual Temperature & Precip. Seasonality       & 0.82                 & 0.36             \\
{\it Betula alleghaniensis}   & Mean Annual Temperature & Mean Annual Precipitation & 1.45                 & 0.35             \\
{\it Betula papyrifera}       & Mean Annual Temperature & Precip. Seasonality       & 0.90                 & 0.32             \\
{\it Fagus grandifolia}       & Mean Annual Temperature & Precip. Warmest Quarter   & 0.52                 & 0.41             \\
{\it Fraxinus americana}      & Mean Annual Temperature & Mean Annual Precipitation & 0.28                 & 0.23             \\
{\it Fraxinus pennsylvanica}  & Mean Annual Temperature & Mean Annual Precipitation & 1.80                 & 0.40             \\
{\it Liriodendron tulipifera} & Mean Annual Temperature &  Precip. Warmest Quarter  & 0.75                 & 0.30             \\
{\it Nyssa sylvatica}         & Mean Temp Wettest Q.    & Precip. Seasonality       & 0.35                 & 0.18             \\
{\it Picea glauca}            & Mean Diurnal Range      & Precip. Seasonality       & 0.64                 & 0.12             \\
{\it Picea mariana}           & Mean Annual Temperature & Precip. Warmest Quarter   & 1.39                 & 0.34             \\
{\it Pinus banksiana}         & Mean Annual Temperature & Mean Annual Precipitation & 0.76                 & 0.22             \\
{\it Pinus strobus}           & Mean Annual Temperature & Precip. Warmest Quarter   & 0.032                & 0.14             \\
{\it Pinus taeda}             & Mean Annual Temperature & Mean Annual Precipitation & 0.99                 & 0.30             \\
{\it Populus tremuloides}     & Mean Annual Temperature & Mean Annual Precipitation & 0.65                 & 0.12             \\ 
{\it Quercus alba}            & Mean Annual Temperature & Precip. Seasonality       & 0.15                 & 0.27             \\
{\it Quercus nigra}           & Mean Diurnal Range      & Precip. Warmest Quarter   & 0.34                 & 0.092            \\
{\it Quercus rubra}           & Mean Annual Temperature & Mean Annual Precipitation & 1.81                 & 0.50             \\
{\it Quercus velutina}        & Mean Annual Temperature & Precip. Warmest Quarter   & 1.11                 & 0.41             \\
{\it Ulmus americana}         & Mean Annual Temperature & Precip. Seasonality       & 0.79                 & 0.13             \\ 
\bottomrule
\end{tabular}
\begin{tablenotes}
\item [1] $\Delta$AIC refers to absolute AIC difference between the selected model and the next best model in the set.
\item [2] AIC$_w$ can be interpreted as the probability that this model is the best model in the set (cite Burnham and Anderson).
\end{tablenotes}
\end{threeparttable}
\end{table}


\begin{table}[tb]
\label{tab:model_selection}
\caption{Model validation statistics}
\begin{tabular}{lllll}
\toprule
Scientific name               & Calibration ROC & Validation ROC & Calibration TSS & Validation TSS \\ 
\midrule
{\it Abies balsamea}          & 0.9770          & 0.9740          & 0.8651         & 0.8545         \\
{\it Acer rubrum}             & 0.9690          & 0.9684          & 0.8929         & 0.8854         \\
{\it Acer saccharum}          & 0.9873          & 0.9876          & 0.9365         & 0.9377         \\
{\it Betula alleghaniensis}   & 0.9886          & 0.9880          & 0.9193         & 0.9208         \\
{\it Betula papyrifera}       & 0.9791          & 0.9780          & 0.8880         & 0.8856         \\
{\it Fagus grandifolia}       & 0.9841          & 0.9865          & 0.9060         & 0.9135         \\
{\it Fraxinus americana}      & 0.9787          & 0.9774          & 0.8817         & 0.8741         \\
{\it Fraxinus pennsylvanica}  & 0.9691          & 0.9740          & 0.8255         & 0.8379         \\
{\it Liriodendron tulipifera} & 0.9925          & 0.9930          & 0.9146         & 0.9162         \\
{\it Nyssa sylvatica}         & 0.9824          & 0.9833          & 0.8674         & 0.8622         \\
{\it Picea glauca}            & 0.9764          & 0.9748          & 0.8443         & 0.8405         \\
{\it Picea mariana}           & 0.9906          & 0.9893          & 0.9208         & 0.9148         \\
{\it Pinus banksiana}         & 0.9940          & 0.9952          & 0.9405         & 0.9466         \\
{\it Pinus strobus}           & 0.9861          & 0.9844          & 0.9091         & 0.8978         \\
{\it Pinus taeda}             & 0.9937          & 0.9937          & 0.9303         & 0.9278         \\
{\it Populus tremuloides}     & 0.9788          & 0.9769          & 0.8888         & 0.8830         \\ 
{\it Quercus alba}            & 0.9901          & 0.9913          & 0.9382         & 0.9454         \\
{\it Quercus nigra}           & 0.9897          & 0.9907          & 0.8931         & 0.8927         \\
{\it Quercus rubra}           & 0.9812          & 0.9807          & 0.9031         & 0.8980         \\
{\it Quercus velutina}        & 0.9860          & 0.9860          & 0.9018         & 0.8940         \\
{\it Ulmus americana}         & 0.9635          & 0.9646          & 0.8223         & 0.8168         \\
\bottomrule
\end{tabular}
\end{table}


\begin{table}[tb]
\label{tab:parameterized_equations}
\caption{Parameterized equations for colonization and extinction rate}
\begin{tabular}{lll}
\toprule
Species                       & Colonization                               & Extinction                                  \\ \midrule
{\it Abies balsamea}          & $x.xx + x.xxT + x.xxT^2 + x.xxP + x.xxP^2$ & $x.xx + x.xxT + x.xxT^2 + x.xxP + x.xxP^2$  \\
{\it Acer rubrum}             & $x.xx + x.xxT + x.xxP + x.xxP^2$           & $x.xx + x.xxT + x.xxT^2 + x.xxP$            \\
{\it Acer saccharum}          & $x.xx + x.xxT + x.xxT^2 + x.xxP$           & $x.xx + x.xxT + x.xxT^2 + x.xxP$            \\
{\it Betula alleghaniensis}   & $x.xx + x.xxT + x.xxT^2 + x.xxP$           & $x.xx + x.xxT + x.xxP$                      \\
{\it Betula papyrifera}       & $x.xx + x.xxT + x.xxT^2 + x.xxP + x.xxP^2$ & $x.xx + x.xxT + x.xxP$                      \\
{\it Fagus grandifolia}       & $x.xx + x.xxT + x.xxT^2 + x.xxP$           & $x.xx + x.xxT + x.xxT^2 + x.xxP + x.xxP^2$  \\
{\it Fraxinus americana}      & $x.xx + x.xxT + x.xxT^2 + x.xxP$           & $x.xx + x.xxT + x.xxT^2 + x.xxP + x.xxP^2$  \\
{\it Fraxinus pennsylvanica}  & $x.xx + x.xxT + x.xxT^2 + x.xxP + x.xxP^2$ & $x.xx + x.xxT + x.xxT^2 + x.xxP + x.xxP^2$  \\
{\it Liriodendron tulipifera} & $x.xx + x.xxT + x.xxT^2 + x.xxP$           & $x.xx + x.xxT + x.xxT^2 + x.xxP$            \\
{\it Nyssa sylvatica}         & $x.xx + x.xxT + x.xxT^2 + x.xxP + x.xxP^2$ & $x.xx + x.xxT + x.xxP + x.xxP^2$            \\
{\it Picea glauca}            & $x.xx + x.xxT + x.xxT^2 + x.xxP + x.xxP^2$ & $x.xx + x.xxT + x.xxP + x.xxP^2$            \\
{\it Picea mariana}           & $x.xx + x.xxT + x.xxP$                     & $x.xx + x.xxT + x.xxT^2 + x.xxP$            \\
{\it Pinus banksiana}         & $x.xx + x.xxT + x.xxP$                     & $x.xx + x.xxT + x.xxT^2 + x.xxP$            \\
{\it Pinus strobus}           & $x.xx + x.xxT + x.xxP$                     & $x.xx + x.xxT + x.xxT^2 + x.xxP + x.xxP^2$  \\
{\it Pinus taeda}             & $x.xx + x.xxT + x.xxT^2 + x.xxP$           & $x.xx + x.xxT + x.xxT^2 + x.xxP$            \\
{\it Populus tremuloides}     & $x.xx + x.xxT + x.xxT^2 + x.xxP$           & $x.xx + x.xxT + x.xxP$                      \\ 
{\it Quercus alba}            & $x.xx + x.xxT + x.xxT^2 + x.xxP$           & $x.xx + x.xxT + x.xxT^2 + x.xxP$            \\
{\it Quercus nigra}           & $x.xx + x.xxT + x.xxP$                     & $x.xx + x.xxT + x.xxT^2 + x.xxP$            \\
{\it Quercus rubra}           & $x.xx + x.xxT + x.xxT^2 + x.xxP + x.xxP^2$ & $x.xx + x.xxT + x.xxT^2 + x.xxP + x.xxP^2$  \\
{\it Quercus velutina}        & $x.xx + x.xxT + x.xxT^2 + x.xxP + x.xxP^2$ & $x.xx + x.xxT + x.xxT^2 + x.xxP$            \\
{\it Ulmus americana}         & $x.xx + x.xxT + x.xxP$                     & $x.xx + x.xxT + x.xxT^2 + x.xxP$            \\ 
\bottomrule
\end{tabular}
\end{table}


