% Please add the following required packages to your document preamble:
% \usepackage{booktabs}
\begin{table}[tb]
\label{tab:species_list}
\caption{List of species studied and the number of observations in the database}
\begin{tabular}{llll}
\toprule
Scientific name               & English name   & Occurrences & Transitions \\ 
\midrule
{\it Abies balsamea}          & Balsam fir     & 18961       &             \\
{\it Acer rubrum}             & Red maple      & 36726       &             \\
{\it Acer saccharum}          & Sugar maple    & 18634       &             \\
{\it Betula alleghaniensis}   & Yellow birch   & 10371       &             \\
{\it Betula papyrifera}       & Paper birch    & 15531       &             \\
{\it Fagus grandifolia}       & American beech & 9619        &             \\
{\it Fraxinus americana}      & White ash      & 10385       &             \\
{\it Fraxinus pennsylvanica}  &                & 7662        &             \\
{\it Liquidambar styraciflua} &                & 14778       &             \\
{\it Liriodendron tulipifera} &                & 11045       &             \\
{\it Nyssa sylvatica}         &                & 9209        &             \\
{\it Picea glauca}            & White spruce   & 8526        &             \\
{\it Pinus banksiana}         & Jack pine      & 3023        &             \\
{\it Pinus strobus}           &                & 8182        &             \\
{\it Pinus taeda}             &                & 16351       &             \\
{\it Populus tremuloides}     & Quaking aspen  & 12736       &             \\ 
{\it Quercus alba}            & White oak      & 17280       &             \\
{\it Quercus nigra}           &                & 8291        &             \\
{\it Quercus rubra}           &                & 14337       &             \\
{\it Quercus velutina}        &                & 9997        &             \\
{\it Ulmus americana}         & American elm   & 11597       &             \\ 
\bottomrule
\end{tabular}
\end{table}

\begin{table}[tb]
\begin{threeparttable}
\label{tab:model_selection}
\caption{Colonization-extinction model selected for each species studied}
\begin{tabular}{llllll}
\toprule
Scientific name               & Temperature Variable    & Precipitation Variable    & Model\tnote{1} & $\Delta$AIC\tnote{2} & AIC$_w$\tnote{3} \\ 
\midrule
{\it Abies balsamea}          & Mean Annual Temperature & Mean Annual Precipitation & $T^2 + P^2$    & 2.15                 & 0.74             \\
{\it Acer rubrum}             &   &   & & \\
{\it Acer saccharum}          & Mean Annual Temperature & Precip. Seasonality       & $T^2 + P$      & 0.82                 & 0.36             \\
{\it Betula alleghaniensis}   &   &   & & \\
{\it Betula papyrifera}       &   &   & & \\
{\it Fagus grandifolia}       &   &   & & \\
{\it Fraxinus americana}      &   &   & & \\
{\it Fraxinus pennsylvanica}  &   &   & & \\
{\it Liquidambar styraciflua} &   &   & & \\
{\it Liriodendron tulipifera} &   &   & & \\
{\it Nyssa sylvatica}         &   &   & & \\
{\it Picea glauca}            &   &   & & \\
{\it Pinus banksiana}         &   &   & & \\
{\it Pinus strobus}           &   &   & & \\
{\it Pinus taeda}             &   &   & & \\
{\it Populus tremuloides}     &   &   & & \\ 
{\it Quercus alba}            &   &   & & \\
{\it Quercus nigra}           &   &   & & \\
{\it Quercus rubra}           &   &   & & \\
{\it Quercus velutina}        &   &   & & \\
{\it Ulmus americana}         &   &   & & \\ 
\bottomrule
\end{tabular}
\begin{tablenotes}
\item [1] Model specification shows whether temperature (T) and precipitation (P) variables were included as well as the order of polynomial used.
\item [2] $\Delta$BIC refers to absolute BIC difference between the selected model and the next best model in the set.
\item [3] BIC$_w$ can be interpreted as the probability that this model is the best model in the set (cite Burnham and Anderson).
\end{tablenotes}
\end{threeparttable}
\end{table}


\begin{table}[tb]
\label{tab:model_selection}
\caption{Model validation statistics}
\begin{tabular}{lllll}
\toprule
Scientific name               & Calibration ROC & Validation ROC & Calibration TSS & Validation TSS \\ 
\midrule
{\it Abies balsamea}          & 0.9770          & 0.9740          & 0.8651         & 0.8545         \\
{\it Acer rubrum}             &   &   &  \\
{\it Acer saccharum}          & 0.9873          & 0.9876          & 0.9365         & 0.9377         \\
{\it Betula alleghaniensis}   &   &   &  \\
{\it Betula papyrifera}       &   &   &  \\
{\it Fagus grandifolia}       &   &   &  \\
{\it Fraxinus americana}      &   &   &  \\
{\it Fraxinus pennsylvanica}  &   &   &  \\
{\it Liquidambar styraciflua} &   &   &  \\
{\it Liriodendron tulipifera} &   &   & & \\
{\it Nyssa sylvatica}         &   &   & & \\
{\it Picea glauca}            &   &   & & \\
{\it Pinus banksiana}         &   &   & & \\
{\it Pinus strobus}           &   &   & & \\
{\it Pinus taeda}             &   &   & & \\
{\it Populus tremuloides}     &   &   & & \\ 
{\it Quercus alba}            &   &   & & \\
{\it Quercus nigra}           &   &   & & \\
{\it Quercus rubra}           &   &   & & \\
{\it Quercus velutina}        &   &   & & \\
{\it Ulmus americana}         &   &   & & \\ 
\bottomrule
\end{tabular}
\end{table}


\begin{table}[tb]
\label{tab:parameterized_equations}
\caption{Parameterized equations for colonization and extinction rate}
\begin{tabular}{lll}
\toprule
Species    & Colonization & Extinction  \\ \midrule
{\it Abies balsamea} & & \\ 
\bottomrule
\end{tabular}
\end{table}


