\documentclass[11pt]{article}

\usepackage[margin=1in]{geometry}

% use proper unicode fonts
\usepackage[T1]{fontenc}
\usepackage[utf8]{inputenc}

\usepackage{amsmath} % for better display of equations

\usepackage{setspace}
%\usepackage{natbib} 

\usepackage{titling} % controls the way the title information is displayed
\pretitle{\begin{flushleft}\Large}
\posttitle{\end{flushleft}}
\predate{}
\postdate{}
\preauthor{\begin{flushleft}}
\postauthor{\end{flushleft}}
\setlength{\droptitle}{-3em}

\usepackage{authblk} % adds some nice options for displaying the author list
\renewcommand\Authsep{\protect\\}
\renewcommand\Authands{\protect\\}

%% graphics packages
\usepackage{graphicx}
%\usepackage[nomarkers, tablesfirst]{endfloat} % for final
\usepackage{caption}
%\captionsetup{labelsep=none,textformat=empty} % for final
\captionsetup{labelformat=simple} % for drafts
\usepackage{booktabs}
\usepackage[flushleft]{threeparttable}


%% ----------------------------------
%
%     Title and authorship information
%
%% ----------------------------------


\title{Colonization-extinction dynamics determine tree species range limits in northeastern North America}
\date{}
\author[1,2,3,4]{Matthew V. Talluto (mtalluto@gmail.com)}
\author[1]{Isabelle Boulangeat (isabelle.boulangeat@gmail.com)}
\author[1]{Steve Vissault (steve.vissault@uqar.ca)}
\author[2,3]{Wilfried Thuiller (wilfried.thuiller@ujf-grenoble.fr)}
\author[1]{Dominique Gravel (dominique\_gravel@uqar.ca)}
\affil[1]{Département de biologie, Université du Québec à Rimouski, Rimouski, Quebec, Canada}
\affil[2]{Université Grenoble Alpes, Laboratoire d’Ecologie Alpine (LECA), F-38000 Grenoble, France}
\affil[3]{CNRS, Laboratoire d’Ecologie Alpine (LECA), F-38000 Grenoble, France}
\affil[4]{Author for correspondance. Address: Departament de Biologie, chimie, et geographie, 300, Allée des Ursulines, Rimouski, Quebec G5L 3A1, Canada}


%% ----------------------------------
%
%     END PREAMBLE
%
%% ----------------------------------

\begin{document}
%\doublespacing
%% ----------------------------------
%
%     TITLE PAGE
%
%% ----------------------------------

%TC:ignore
\begin{titlingpage}
	\maketitle
	
	\begin{flushleft}
	
	\textbf{Short title:} Colonization-extinction dynamics of eastern North American trees
		
	\textbf{Keywords:}
	\end{flushleft}
\end{titlingpage}
%TC:endignore

%TC:break abstract
\begin{abstract}
\noindent

 \end{abstract}

%TC:break manuscript
%% ----------------------------------
%
%     INTRODUCTION
%
%% ----------------------------------

\section*{Introduction}
\subsection*{Objectives}
\begin{itemize}
	\item Describe the range limits at equilibrium of ecologically important trees in eastern North America as predicted by metapopulation theory
	\item Evaluate the relationship between macroclimatic variables and colonization and extinction rates
	\item Test quantitatively whether the equilibrium prediction matches the contemporary observed range
	\item Compare among species the predicted resilience to environmental change 
\end{itemize}

Present and future rapidly changing global environmental conditions are likely to push many species to the limits of their ecological niches.
Biogeographical theory and empirical work suggests that species can respond to these changes by adapting \emph{in situ} to the new conditions, geographically tracking ideal conditions (resulting in range shifts), and/or experiencing local extinction where conditions no longer support presence.
Species distribution models (SDMs) are commonly-applied to understand the latter two options.
Classical SDMs often use a bioclimatic modeling approach that statistically relates presence or absence of the species to climate.
Such models are quite powerful when describing the present realized niche of a species, however they suffer from several drawbacks when predicting the future distribution of species, and they provide no insight into the mechanisms underlying a species' response to the environment.
An alternative more mechanistic approach is to examine the colonization rate $c$ of unoccupied patches and the extinction rate $e$ of occupied patches.
(here maybe briefly review the literature on the subject)
In such a stochastic system, the long-term expectation is that presence will only be possible where $c > e$.
It follows that, if these rates vary along an environmental gradient, the equilibrium position of the range limit on the gradient will be the point at which $c = e$.

We built a dynamic state-and-transition model (STM) using classic metapopulation theory \citep{Levins1969, Nee1992} and the foundation.
The classical formulation of this model is inappropriate for modeling range dynamics at large scales, as it assumes colonization and extinction rates are constant in space and time and treats prevalence as a global variable.
Holt and Keitt extended the Levins model to a scale appropriate for range dynamics by varying the parameters along an environmental gradient, showing that, under metapopulation dynamics, range limits of a theoretical species arise when an environmental gradient causes an increase in the extinction rate, a decrease in the colonization rate, or a decrease in available habitat \cite{Holt2000}.

We applied a model similar to the Holt model \cite{Holt2000} to 21 forest tree species that are abundant in eastern North American temperate and boreal forests.
Our model, which is discrete in space and time, treats the landscape as a series of patches, where the state $Y_{x,t}$ of patch $x$ at time $t$ is a function of the colonization ($c_{x,t}$) and extinction ($e_{x,t}$) rates, as well as the prevalence of the species ($phi_{x,t}$).
Both the colonization and extinction rates were parameterized as a function of two environmental variables, one related to temperature and one precipitation, thus $c_x = f(T_x, P_x); e_x = g(T_X, P_x)$ (see \emph{Methods} for descriptions of the specific variables used).
As our objectives concern equilibrial dynamics and the role of CE dynamics in determining species ranges, for the present we ignore temporal dynamics and focus instead on spatial variation in the model parameters.

Our model is a first order Markov chain that is dynamic in space and time (though we fix the time dimension for this paper). The state of a patch depends on the state of local patches at the previous time step and the transition rates. Range boundaries emerge at equilibrium due to the effect of environmental gradients on the transition rates

We apply our model exclusively to forest trees, and because our model is parameterized using forest plot data, we treat all patches as potential habitat, thus setting $k=1$ from the Levins model.
Following this simplification, the equilibrium condition for the presence of a species on the environmental gradient is any point $x$ such that $c_x > e_x$ \citep{Holt2000}.
It follows that the range limit of the species will occur where $c_x = e_x$.


We designed a colonization-extinction model with a dynamic parameterization that allowed the colonization and extinction rates to vary as a function of climate.
A major challenge to employing these models is the lack of time series data needed to estimate the parameters.
We used a large database of permanent forest plots, containing repeated observations of xxx plots in eastern North America (n=xxx paired observations). 
We selected the 21 most abundant species (based on the number of observations in the database) as the focus of this study.
We performed model selection, using 6 macroclimatic variables, then parameterized the model using Markov Chain Monte Carlo (MCMC) sampling and addressed the following questions for each selected species:
\begin{enumerate}
	\item Are range limits driven by colonization-extinction dynamics?
	\item To what extent are the species studied presently out of equilibrium with the environment?
	\item How resilient are range limits to perturbations in the environment?
	\item Which climatic variables best describe colonization and extinction rates?
\end{enumerate}

%% here expand a bit on each question above and perhaps provide a testable hypothesis.
We evaluated the model by reserving half of the observed sample pairs.
We parameterized the model and performed model selection on the calibration half, then used the parameters to predict the second state as a function of the climate and the first state in the evaluation dataset.
We then compared the predicted to the observed second state and computed TSS and ROC (explanation and citation?).
This evaluation can be interpreted as a test of the model iteself, and thus of the hypothesis that range limits are determined by colonization-extinction dynamics.

To test to what extent species were out of equilibrium with the environment, we employed an independent dataset from the same database.
This dataset consisted of plots that were only sampled once, and thus were not useful in parameterizing the CE model.
We used the posterior predictions from the CE model to compute the probability of presence at equilibrium; we then compared this with observed presences from the dataset and computed TSS and ROC.
For comparison with a highly fit model, we also built an SDM using the Random Forests algorithm and computed the same statistics.
Species that are in equilibrium with their environment will show high coherence between the two models, and thus should have high validation scores in for both the CE and SDM.
Species out of equilibrium should perform poorly for the CE model and better for the SDM.


\section*{Results}
\begin{itemize}
	\item response of C-E to env variables
	\item validation of models w.r.t. reserved data (how good is the model)
	\item ability of the model to predict the range (equilibrium)
	\item rate of response to disequilibrium
	\item ability of model to predict response to environmental change
\end{itemize}
\begin{figure}
	\includegraphics[width=5in]{../img/response_curves.pdf}
	\caption{
		Response of colonization (blue) and extinction (red) rates of selected species to the environment.
		Solid lines are the median predictions of xx posterior samples, and shaded regions show 95\% Bayesian credible intervals.
		Species not shown here can be found in supporting information.
	}
	\label{fig:response_curves}
\end{figure}


\begin{figure}
	\includegraphics[width=6.5in]{../img/posterior_maps.pdf}
	\caption{
		Maps showing the predicted colonization rate (blue, left column), extinction rate(red, second column), posterior probability of presence from the CE model (third column), and predicted probability of presence from a conventional SDM (right column).
		Posterior probability of presence is computed as the proportion of posterior samples where colonization exceeds extinction for a given location.
		Species not shown here can be found in supporting information.
	}
	\label{fig:response_curves}
\end{figure}


\section*{Discussion}

Fisichelli and colleagues reported recruitement of temperate trees into boreal ecosystems, supporting our conclusion that colonization of temperature trees varies as a function of temperature \cite{Fisichelli2014}.



This model enables us to more precisely define the niche in terms of the fundamental processes of colonization and extinction (fig..). 
For example, the response of \emph(Acer saccharum) to temperature suggests that the southern range limit is a consequence of elevated extinction rates, whereas the northern limit is likely determined by low rates of colonization.
The consequence of these two processes, when overlaid, is a range covering intermediate temperatures, bounded by low colonization in the north and high extinction in the south (fig - map).
A further advantage of the model, particularly when compared to static correlative SDMs, is the ability to investigate the resilience of the system at the range margin.
At the range boundary, the dominant eigenvalue, representing the response of the system to small perturbations (i.e., recovery after disturbance or the acquisition of the new equilibrium following climate change) approaches zero \cite{Scheffer2009}.
The rate at which this value approaches zero (i.e., the relative slopes of $c$ and $e$ along the environmental gradient -- ref response curves) indicates the degree to which the range margin will respond to changes in the gradient.
Thus, "fast" species with steep slopes along the gradient (e.g., list some species) will respond quickly to climate change, while "slow" species may lag behind climate change, producing instability and potentially promoting catastrophic changes (e.g., widespread and rapid local extinctions along the range margin) \cite{Scheffer2009, Scheffer2012}. %% NOTE - this may be a paper in itself - worth considering. It could be cool to actually measure this and then investigate clim change scenarios via simulating this model. could easily measure both the variance and the autocorrelation in such sims

% can this apply to other organisms? animals prob not - dispersal is just too high
% this deserves a paragraph here
Our model is particularly appropriate for forest trees, which are sessile, disperse relatively short distances, and are long-lived. 
Unique challenges confront applying such a model without these characteristics, such as most animals. 
Animals in highly patchy environments, with local populations linked by moderate dispersal rates, are likely to be the most appropriate species for use with our model.

\subsection*{Conclusions}
\begin{itemize}
	\item Range limits of [some/many/most/all] [temperate/boreal] tree species in northeastern north America are determined by colonization-extinction dynamics
	\item Moisture and energy (i.e., mean annual temperature and precipitation) are associated with colonization and extinction rates for many species
	\item For some species, the equilibrium state of the model did not match the predicted distribution of the species, suggesting that these species are either out of equilibrium with the environment, or that other processes (e.g., large-scale disturbances) contribute to their range dynamics.
	\item Resilience (represented by the dominant eigenvalue (c - e) near the transition point) varies widely among species, suggesting greatly differential rates of response to environmental change.
	\item Validation? Model Selection?
\end{itemize}


\subsection*{List of figures to make}
\begin{itemize}
	\item Fig 3: maps for selected species (with remaining in appendix) - 4 panels - c; e; predicted range; SDM
	\item Fig 3: eigenvalue maps
\end{itemize}

\subsection*{Tables}
% Please add the following required packages to your document preamble:
% \usepackage{booktabs}
\begin{table}[tb]
\label{tab:species_list}
\caption{List of species studied and the number of observations in the database}
\begin{tabular}{llll}
\toprule
Scientific name               & English name   & Occurrences & Transitions \\ 
\midrule
{\it Abies balsamea}          & Balsam fir     & 18961       &             \\
{\it Acer rubrum}             & Red maple      & 36726       &             \\
{\it Acer saccharum}          & Sugar maple    & 18634       &             \\
{\it Betula alleghaniensis}   & Yellow birch   & 10371       &             \\
{\it Betula papyrifera}       & Paper birch    & 15531       &             \\
{\it Fagus grandifolia}       & American beech & 9619        &             \\
{\it Fraxinus americana}      & White ash      & 10385       &             \\
{\it Fraxinus pennsylvanica}  &                & 7662        &             \\
{\it Liquidambar styraciflua} &                & 14778       &             \\
{\it Liriodendron tulipifera} &                & 11045       &             \\
{\it Nyssa sylvatica}         &                & 9209        &             \\
{\it Picea glauca}            & White spruce   & 8526        &             \\
{\it Pinus banksiana}         & Jack pine      & 3023        &             \\
{\it Pinus strobus}           &                & 8182        &             \\
{\it Pinus taeda}             &                & 16351       &             \\
{\it Populus tremuloides}     & Quaking aspen  & 12736       &             \\ 
{\it Quercus alba}            & White oak      & 17280       &             \\
{\it Quercus nigra}           &                & 8291        &             \\
{\it Quercus rubra}           &                & 14337       &             \\
{\it Quercus velutina}        &                & 9997        &             \\
{\it Ulmus americana}         & American elm   & 11597       &             \\ 
\bottomrule
\end{tabular}
\end{table}

\begin{table}[tb]
\begin{threeparttable}
\label{tab:model_selection}
\caption{Colonization-extinction model selected for each species studied}
\begin{tabular}{llllll}
\toprule
Scientific name               & Temperature Variable    & Precipitation Variable    & Model\tnote{1} & $\Delta$AIC\tnote{2} & AIC$_w$\tnote{3} \\ 
\midrule
{\it Abies balsamea}          & Mean Annual Temperature & Mean Annual Precipitation & $T^2 + P^2$    & 2.15                 & 0.74             \\
{\it Acer rubrum}             &   &   & & \\
{\it Acer saccharum}          & Mean Annual Temperature & Precip. Seasonality       & $T^2 + P$      & 0.82                 & 0.36             \\
{\it Betula alleghaniensis}   &   &   & & \\
{\it Betula papyrifera}       &   &   & & \\
{\it Fagus grandifolia}       &   &   & & \\
{\it Fraxinus americana}      &   &   & & \\
{\it Fraxinus pennsylvanica}  &   &   & & \\
{\it Liquidambar styraciflua} &   &   & & \\
{\it Liriodendron tulipifera} &   &   & & \\
{\it Nyssa sylvatica}         &   &   & & \\
{\it Picea glauca}            &   &   & & \\
{\it Pinus banksiana}         &   &   & & \\
{\it Pinus strobus}           &   &   & & \\
{\it Pinus taeda}             &   &   & & \\
{\it Populus tremuloides}     &   &   & & \\ 
{\it Quercus alba}            &   &   & & \\
{\it Quercus nigra}           &   &   & & \\
{\it Quercus rubra}           &   &   & & \\
{\it Quercus velutina}        &   &   & & \\
{\it Ulmus americana}         &   &   & & \\ 
\bottomrule
\end{tabular}
\begin{tablenotes}
\item [1] Model specification shows whether temperature (T) and precipitation (P) variables were included as well as the order of polynomial used.
\item [2] $\Delta$BIC refers to absolute BIC difference between the selected model and the next best model in the set.
\item [3] BIC$_w$ can be interpreted as the probability that this model is the best model in the set (cite Burnham and Anderson).
\end{tablenotes}
\end{threeparttable}
\end{table}


\begin{table}[tb]
\label{tab:model_selection}
\caption{Model validation statistics}
\begin{tabular}{lllll}
\toprule
Scientific name               & Calibration ROC & Validation ROC & Calibration TSS & Validation TSS \\ 
\midrule
{\it Abies balsamea}          & 0.9770          & 0.9740          & 0.8651         & 0.8545         \\
{\it Acer rubrum}             &   &   &  \\
{\it Acer saccharum}          & 0.9873          & 0.9876          & 0.9365         & 0.9377         \\
{\it Betula alleghaniensis}   &   &   &  \\
{\it Betula papyrifera}       &   &   &  \\
{\it Fagus grandifolia}       &   &   &  \\
{\it Fraxinus americana}      &   &   &  \\
{\it Fraxinus pennsylvanica}  &   &   &  \\
{\it Liquidambar styraciflua} &   &   &  \\
{\it Liriodendron tulipifera} &   &   & & \\
{\it Nyssa sylvatica}         &   &   & & \\
{\it Picea glauca}            &   &   & & \\
{\it Pinus banksiana}         &   &   & & \\
{\it Pinus strobus}           &   &   & & \\
{\it Pinus taeda}             &   &   & & \\
{\it Populus tremuloides}     &   &   & & \\ 
{\it Quercus alba}            &   &   & & \\
{\it Quercus nigra}           &   &   & & \\
{\it Quercus rubra}           &   &   & & \\
{\it Quercus velutina}        &   &   & & \\
{\it Ulmus americana}         &   &   & & \\ 
\bottomrule
\end{tabular}
\end{table}


\begin{table}[tb]
\label{tab:parameterized_equations}
\caption{Parameterized equations for colonization and extinction rate}
\begin{tabular}{lll}
\toprule
Species    & Colonization & Extinction  \\ \midrule
{\it Abies balsamea} & & \\ 
\bottomrule
\end{tabular}
\end{table}





\section*{Methods}

\subsection*{Model description}

\subsection*{Prevalence model}
We used the RF package in R to build models of regional prevalence for each species. 
For predictors, we initially evaluated a set of xx environmental variables. 
From these, we selected xx variables that were relatively uncorrelated to each other (pairwise Pearson correlation < xx) and that had high loadings in a principal components analysis (PCA) of all variables. 
In this way, we reduced the risk of estimation problems due to correlations among the predictors while retaining variables that were representative of the available climatic space.

\subsection*{Model selection}
For each species, we evaluated a set of 145 colonization-extinction models. 
Each model consisted of at most two environmental variables, one associated with temperature and the other precipitation\. 
The model set was constructed from eight possible model forms (Table?) applied to all possible two-variable combinations of the four energy and four moisture variables (i.e., 16 possible variable combinations), along with two additional single-variable forms for each variable and an intercept-only model. 
Because of the large number of models, fitting all models using MCMC was infeasible; therefore, we performed model selection by computing the Akaike Information Criterion (AIC) on maximum likelihood estimates for each model, fit using simulated annealing (package GenSA in R). 
We then selected the model that minimized AIC for final parameterization.

\subsection*{Parameterization}
We used a Markov Chain Monte Carlo (MCMC) sampler employing a Metropolis-Hastings scheme to estimate posterior distributions for all parameters.
This method has a number of advantages over the simulated annealing approach used for preliminary model selection.
First, MCMC results in full joint posterior distributions of the parameters, allowing for a robust understanding of the uncertainty both in model parameters and predictions.
Second, the use of a Bayesian framework allows the inclusion of prior information.
As our parameters were all estimated on the logit scale, we used the weakly informative Cauchy priors suggested by Gelman and colleagues \cite{Gelman2008}, with a center of 0 and scale of 10 for intercept parameters or 2.5 for slope parameters.
Finally, having posterior distributions of the predictors allows for the information produced by our model to be incorporated in future meta-modeling efforts that can further improve our characterization of species' ranges and our understanding of model uncertainty (cite GEB paper).

For each species, we obtained xxx posterior samples...

We assessed convergence using the multivariate potential scale reduction factor (MPSR) \cite{Brooks1998}.
This method compares the variance within and among multiple parallel MCMC chains, where a value of 1 indicates perfect convergence (i.e., the chains are from indistinguishable distributions), and values approximately \textgreater1.1 indicate a lack of convergence.

For climate variables, we used regional climate models at xx resolution \cite{McKenney2011}.

\renewcommand\refname{Literature Cited}
\bibliography{2state}{}
\bibliographystyle{pnas}

\end{document}
